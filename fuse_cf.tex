
\documentclass[conference]{IEEEtran}


% *** GRAPHICS RELATED PACKAGES ***
%
\ifCLASSINFOpdf
  % \usepackage[pdftex]{graphicx}
  % declare the path(s) where your graphic files are
  % \graphicspath{{../pdf/}{../jpeg/}}
  % and their extensions so you won't have to specify these with
  % every instance of \includegraphics
  % \DeclareGraphicsExtensions{.pdf,.jpeg,.png}
\else
  % or other class option (dvipsone, dvipdf, if not using dvips). graphicx
  % will default to the driver specified in the system graphics.cfg if no
  % driver is specified.
  % \usepackage[dvips]{graphicx}
  % declare the path(s) where your graphic files are
  % \graphicspath{{../eps/}}
  % and their extensions so you won't have to specify these with
  % every instance of \includegraphics
  % \DeclareGraphicsExtensions{.eps}
\fi
% graphicx was written by David Carlisle and Sebastian Rahtz. It is
% required if you want graphics, photos, etc. graphicx.sty is already
% installed on most LaTeX systems. The latest version and documentation can
% be obtained at: 
% http://www.ctan.org/tex-archive/macros/latex/required/graphics/
% Another good source of documentation is "Using Imported Graphics in
% LaTeX2e" by Keith Reckdahl which can be found as epslatex.ps or
% epslatex.pdf at: http://www.ctan.org/tex-archive/info/
%
% latex, and pdflatex in dvi mode, support graphics in encapsulated
% postscript (.eps) format. pdflatex in pdf mode supports graphics
% in .pdf, .jpeg, .png and .mps (metapost) formats. Users should ensure
% that all non-photo figures use a vector format (.eps, .pdf, .mps) and
% not a bitmapped formats (.jpeg, .png). IEEE frowns on bitmapped formats
% which can result in "jaggedy"/blurry rendering of lines and letters as
% well as large increases in file sizes.
%
% You can find documentation about the pdfTeX application at:
% http://www.tug.org/applications/pdftex

% correct bad hyphenation here
\hyphenation{}

\usepackage{cite}


\begin{document}
%
% paper title
% can use linebreaks \\ within to get better formatting as desired
\title{Recommending Functions in Spreadsheets from the Fuse Corpus}


% author names and affiliations
% use a multiple column layout for up to three different
% affiliations
\author{\IEEEauthorblockN{Shaown Sarker, Emerson Murphy-Hill}
\IEEEauthorblockA{Department of Computer Science\\
NC State University\\
Raleigh, North Carolina\\
Email: \{ssarker, emurph3\}@ncsu.edu}
}

% conference papers do not typically use \thanks and this command
% is locked out in conference mode. If really needed, such as for
% the acknowledgment of grants, issue a \IEEEoverridecommandlockouts
% after \documentclass

% for over three affiliations, or if they all won't fit within the width
% of the page, use this alternative format:
% 
%\author{\IEEEauthorblockN{Michael Shell\IEEEauthorrefmark{1},
%Homer Simpson\IEEEauthorrefmark{2},
%James Kirk\IEEEauthorrefmark{3}, 
%Montgomery Scott\IEEEauthorrefmark{3} and
%Eldon Tyrell\IEEEauthorrefmark{4}}
%\IEEEauthorblockA{\IEEEauthorrefmark{1}School of Electrical and Computer Engineering\\
%Georgia Institute of Technology,
%Atlanta, Georgia 30332--0250\\ Email: see http://www.michaelshell.org/contact.html}
%\IEEEauthorblockA{\IEEEauthorrefmark{2}Twentieth Century Fox, Springfield, USA\\
%Email: homer@thesimpsons.com}
%\IEEEauthorblockA{\IEEEauthorrefmark{3}Starfleet Academy, San Francisco, California 96678-2391\\
%Telephone: (800) 555--1212, Fax: (888) 555--1212}
%\IEEEauthorblockA{\IEEEauthorrefmark{4}Tyrell Inc., 123 Replicant Street, Los Angeles, California 90210--4321}}


% use for special paper notices
%\IEEEspecialpapernotice{(Invited Paper)}

% make the title area
\maketitle


\begin{abstract}
Spreadsheets are the most common form of end-user programming software. Although spreadsheets have a large array of functions built-in, spreadsheet users tend to ignore using them to perform their tasks. To address this issue, we investigate recommender system technologies and consider two distinct approaches to a function recommender system for spreadsheets. In this paper, we use collaborative filtering and the most popular algorithm to recommend functions to spreadsheet users. We apply these algorithms on the Fuse spreadsheet corpus to produce personalized function recommendations to an individual spreadsheet user. Our automated evaluation shows that the collaborative filtering based approach outperforms the most popular algorithm by \texttt{percentage obtained from future evaluation}. Although recommendation in spreadsheets can be difficult compared to other software applications, the results suggest that we can still have useful function recommendations based on the users' usage history.
\end{abstract}


% For peer review papers, you can put extra information on the cover
% page as needed:
% \ifCLASSOPTIONpeerreview
% \begin{center} \bfseries EDICS Category: 3-BBND \end{center}
% \fi
%
% For peerreview papers, this IEEEtran command inserts a page break and
% creates the second title. It will be ignored for other modes.
\IEEEpeerreviewmaketitle

\section{Introduction}

% The spreadsheet users are important.
By definition, end-user programmers are people who are not professional software developers, but makes use of tools and processes that lets them perform tasks similar to programming~\cite{ko2011state}. According to a study from 2005~\cite{scaffidi2005estimating}, nearly 23 million Americans use spreadsheets, constituting 30\% of the entire workforce. Spreadsheets are also commonly used for analytical purposes in industry, almost 90\% of all analysts use spreadsheets to perform their calculations~\cite{winston2001executive}. Based on their number, spreadsheet users form the largest demographic within end-user programmers.

 From a spreadsheet user's perspective, formulas can be viewed as program fragments as formulas can contain programming constructs such as variables, conditional statements, and references to other parts of the spreadsheet. It is important that these end-user programmers using spreadsheet makes efficient use of the functionality available to them \texttt{[Citation needed?]}.

In Microsoft Excel, there are almost 350 unique functions, (472 in Microsoft Excel 2013 including the compatibility functions\footnote{https://support.office.com/en-us/article/Excel-functions-by-category-5f91f4e9-7b42-46d2-9bd1-63f26a86c0eb}). But, most of these are rarely used \texttt{[Citation needed?]}. To facilitate these very large number of spreadsheet analysts to use these functions, it is viable that we explore effective function recommendation in spreadsheets.

Researchers have been assembling and preserving spreadsheet corpora for the purpose of better understanding of end-user activities and designing tools to assist them~\cite{fisher2005euses, hermans2014enron}. In this paper, we look into Fuse~\cite{barik2015fuse}, the largest reproducible spreadsheet corpus known to date. We applied a slightly modified version of the collaborative filtering algorithm on the spreadsheets in Fuse to get function recommendations for an input spreadsheet. We evaluated the performance of our system against another commonly used recommender algorithm and found out that our collaborative filtering-based recommendation system recommends functions in spreadsheet by \texttt{insert future calculated percentage here}. 

The remainder of the paper is structured as follows. We first review previous work on recommendation systems in softwares and related work on making spreadsheet formula usage efficient (Section II). We then provide details of the spreadsheet corpus used for our system and the details of the algorithm used (Section III) before describing our evaluation measures (Section IV) and their outcome (Section V). We discuss the various aspects of our system (Section VI) along with the limitations and future works (Section VII) based on our contribution in this paper afterwards.

\section{Related Work}
Recommender systems are used generally to produce a list of predictions based on the preference of an user and the similarity of preferences of other users. Recommender systems assist users by recommending  products and services to her based on her past preference in a scenario where the number of available options are extremely large. In recent years, recommender systems have become quite popular and have been applied successfully in multiple sectors of business and academia~\cite{hill1995recommending, resnick1994grouplens,shardanand1995social}. 

Collaborative filtering is one of the most prominent recommending technologies available. The core concept of collaborative filtering is to apply a nearest neighbor method between a user's preferences and the preference of a large user community, and provide the user with recommendations by extrapolating based on how her selection or preference relate to that of the community. Collaborative filtering has been applied to recommend a fringe of products and services, ranging from movies to school courses~\cite{miller2003movielens, resnick1994grouplens, linden2003amazon, liu2003adaptive, mcnee2006don, farzan2006social, hsu2008personalized}. 

However, there has been little effort to use recommendation systems to help users of a large   software system with vast set of functionality to learn these functionalities. One of the notable attempts to use recommender systems to recommend command is the OWL~\cite{linton2000owl}, which makes command recommendations to Microsoft Word users. OWL recommends commands to an individual if she is not using certain commands at all but the community is using them on a frequent basis.

Matejka and colleagues developed a collaborative filtering-based approach to recommend commands in AutoCAD, a computer aided drafting software, called CommunityCommands~\cite{matejka2009communitycommands}. The researchers of CommunityCommands showed that the collaborative filtering derived command recommendations were preferred by the AutoCAD users over the history based recommendation approach used in OWL.

In this paper, we build on these prior works by applying the concept of recommender systems to recommend functions in spreadsheets. We use the Fuse spreadsheet corpus as a source of function usage preference of the spreadsheet users' community and recommend functions contextualized to an individual user by applying collaborative filtering algorithm.

\section{Methodology}
According to the formal definition, collaborative filtering is a method of making automatic predictions (filtering) about the interests of a user by collecting preferences or taste information from many users (collaborating) \footnote{https://en.wikipedia.org/wiki/Collaborative\_filtering}. Although collaborative filtering can be applied in many ways, most approaches falling under the category of user-based collaborative filtering are composed of mainly two distinct steps - find the users with similar preference patterns to that of the input user and then use the preference of the similar users to obtain predictions for the input user.

To apply this technique, we need to measure the similarity between two users. In order to do so, we define a vector for each user's spreadsheets and compare the vectors to find the similar preferences. Our function vector is defined such that the vector $V$ for an individual user consists of cells, where each cell $V_i$ represents the usage frequency of the function $f_i$.

To generate the input user's vector we extract function usage frequencies from multiple files created by the user. The more spreadsheets created by the same user is given as input the better representative vector for the user we get. As for the potential pool of similar user vectors, we decided to use Fuse, a spreadsheet corpus containing nearly 250 thousand unique spreadsheets extracted from the Common Crawl\footnote{http://www.commoncrawl.org} index~\cite{barik2015fuse}. The authors of Fuse also extracted meta information such as created by and function usage frequencies. We classified the spreadsheet meta information and discarded any spreadsheet that did not use any function. This reduced the number of spreadsheets under consideration to less than 13 thousand. These spreadsheets were then grouped by the user name \texttt{(There are a large number of spreadsheets where this information is not present, how do we refer to them in the paper?)}. We proceeded to extract the function vector from each of these groups in the similar way the input user vector was extracted. The final number of function vectors in this pool of potential similar user vectors was \texttt{insert actual number from modified code here}.

To measure the similarity between users, we used the cosine similarity function, which measures the cosine of the angle between the user's vectors as described above. Given the function vector for two users $a$ and $b$ as $V_a$ and $V_b$ respectively, the similarity function $s$ is given by:

\begin{center}
	\[
	s(a, b) = \cos({}_{V_a}\theta_{V_b}) = \frac{V_a \cdot V_b}{||V_a|| ||V_b||} = \frac{\sum\limits_{k=1}^{n} V_{a_k}V_{b_k}}{\sqrt{\sum\limits_{k=1}^{n} {V_{a_k}}^2  {V_{b_k}}^2}}
	\]
\end{center}

When the similarity function evaluates closer to 0, the vectors are substantially orthogonal to each other, which indicates that the users are dissimilar. A similarity function value of closer to 1, indicates that the vectors are nearly collinear and the users are quite similar. For ease of our implementation we used the complement of the cosine similarity: $1 - similarity(i,j)$, which is often commonly referred to as \textit{cosine distance}.

To find the similar user vectors for the given vector, we calculated the similarity function value between each of the vector in the pool and the input vector, after this the vectors in the pool were sorted based on their cosine distance in ascending order. We selected $n$ most similar vectors from this ordered list, where $n$ is a tuning parameter, which in our case was between 3 and 5.

We compared the frequencies of each individual function between each of the similar vectors and the input vector, and added a function to the recommendation set only if the function had a frequency value in one of the similar vectors but not in the input vector. The set of recommended functions was then ordered by the cumulative frequency of the functions in the similar vectors, this way the functions in the recommendation list appear in order of their usage in the similar vectors.

\section{Evaluation}
For evaluating the system implemented based on the algorithm in the previous section, we used a cross validation~\cite{Kohavi95astudy} technique, which enabled us to evaluate our system using the existing feature vectors extracted from the Fuse corpus. Cross validation is a model evaluation technique, where the input data is partitioned into two complementary subsets - one subset is used to perform the analysis (training set) and the other is used to validate the analysis on the training set (testing set). Generally, multiple passes of cross validation are performed using different partitions  to cope with variance in the dataset and the results are averaged.

We used a variation of cross validation, called Leave p-out (LPO) cross validation~\cite{arlot2010survey}. According to LPO, all possible training/testing partitions are generated by removing $p$ samples from the complete set. The testing set consists of the $p$ samples and the training set contains the rest $n-p$ samples, where $n$ is the size of the original dataset. Thus, a dataset of length $n$ will have $n \choose p$ possible partitions in LPO cross validation.

We used the function vectors extracted from Fuse to perform LPO cross validation. In LPO cross validation, such a function vector $V_i$ will be partitioned into the training set $S_{training}$ and the testing set $S_{testing}$, where the size of $S_{testing}$ is $p$. The $S_{training}$ set with the rest of the function frequencies set to 0, is used as an input vector for the collaborative filtering-based recommender system, which produces the recommendation set $R_S$. We define the correct number of recommendations for this partition as the number of functions that are both in set $R_S$ and $S_{testing}$ and calculate the result of the LPO evaluation for this function vector $h_i$ as:

\begin{center}
	\[
	h_i = \frac{\sum\limits_{k=1}^{n} |R_{S_k} \cap S_{{testing}_k}|}{n}
	\]
\end{center}

Where $k$ presents each distinct partition and $n$ is $|V_i|\choose p$. We calculated $h_i$ for every vector $V_i$ in the function vector pool with a $p$ value of 3.

\section{Results}
To compare the performance of the collaborative filtering-based recommender system, we also implemented the most popular algorithm~\cite{linton2000owl}. The most popular algorithm recommends functions that are most widely used in the Fuse corpus, but not used by the active user. We used the LPO cross validation to evaluate the recommendations generated by this algorithm and compared the results. Our results show that the collaborative filtering-based system correctly recommends functions for \texttt{insert future calculated number here}  users compared to \texttt{insert future calculated number here} correct recommendations from the most popular algorithm.

\texttt{Graphic comparing the two algorithms. Bar charts?}

\section{Discussion}
\texttt{Consider naming it to contribution and/or merging it with Results section}
\texttt{Any future tool related discussions here, if any.}

\section{Limitations \& Future Work}
From the results presented here, there are a number of attributes of system that provides opportunities for future work.

One of the limitation of the recommendation system was that the pool of potential similar function vectors was small, given that only approximately 5\% of the Fuse spreadsheets used formulas. This however can be alleviated considerably by incorporating other existing spreadsheet corpus like Enron~\cite{hermans2014enron} and EUSES~\cite{fisher2005euses} into the user function vectors pool.

We also made an explicit assumption that the active user's spreadsheets will result in a function vector with some non-zero frequencies. If the active user's spreadsheets are simple data dumps without any function usage, the recommender system can't produce viable recommendations as the similar user vectors in this case will be the ones with lowest function frequency values.

Spreadsheet files, being static, don't contain any temporal or sequential information regarding function discovery. Given this information, it will be possible to recommend more contextualized and personalized functions for the spreadsheet users.

Although we used an automated method to evaluate our system, it cannot replace the recommendations being evaluated by real life spreadsheet users. This type of evaluation can prove how much useful the recommendations actually are. It is possible to conduct a survey, either web-based or real life, where we generate recommendations based on spreadsheet users in the industry and have the actual owners of the spreadsheets evaluate their personalized function recommendations.

\section*{Acknowledgment}
\texttt{Acknowledgements if any (LAS information maybe?)}


% trigger a \newpage just before the given reference
% number - used to balance the columns on the last page
% adjust value as needed - may need to be readjusted if
% the document is modified later
%\IEEEtriggeratref{8}
% The "triggered" command can be changed if desired:
%\IEEEtriggercmd{\enlargethispage{-5in}}

% references section

% can use a bibliography generated by BibTeX as a .bbl file
% BibTeX documentation can be easily obtained at:
% http://www.ctan.org/tex-archive/biblio/bibtex/contrib/doc/
% The IEEEtran BibTeX style support page is at:
% http://www.michaelshell.org/tex/ieeetran/bibtex/
\bibliographystyle{IEEEtran}
% argument is your BibTeX string definitions and bibliography database(s)
\bibliography{fuse_cf}
%
% <OR> manually copy in the resultant .bbl file
% set second argument of \begin to the number of references
% (used to reserve space for the reference number labels box)





% that's all folks
\end{document}
