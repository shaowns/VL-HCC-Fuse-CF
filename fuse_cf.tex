
\documentclass[conference]{IEEEtran}


% *** GRAPHICS RELATED PACKAGES ***
%
\ifCLASSINFOpdf
  % \usepackage[pdftex]{graphicx}
  % declare the path(s) where your graphic files are
  % \graphicspath{{../pdf/}{../jpeg/}}
  % and their extensions so you won't have to specify these with
  % every instance of \includegraphics
  % \DeclareGraphicsExtensions{.pdf,.jpeg,.png}
\else
  % or other class option (dvipsone, dvipdf, if not using dvips). graphicx
  % will default to the driver specified in the system graphics.cfg if no
  % driver is specified.
  % \usepackage[dvips]{graphicx}
  % declare the path(s) where your graphic files are
  % \graphicspath{{../eps/}}
  % and their extensions so you won't have to specify these with
  % every instance of \includegraphics
  % \DeclareGraphicsExtensions{.eps}
\fi
% graphicx was written by David Carlisle and Sebastian Rahtz. It is
% required if you want graphics, photos, etc. graphicx.sty is already
% installed on most LaTeX systems. The latest version and documentation can
% be obtained at: 
% http://www.ctan.org/tex-archive/macros/latex/required/graphics/
% Another good source of documentation is "Using Imported Graphics in
% LaTeX2e" by Keith Reckdahl which can be found as epslatex.ps or
% epslatex.pdf at: http://www.ctan.org/tex-archive/info/
%
% latex, and pdflatex in dvi mode, support graphics in encapsulated
% postscript (.eps) format. pdflatex in pdf mode supports graphics
% in .pdf, .jpeg, .png and .mps (metapost) formats. Users should ensure
% that all non-photo figures use a vector format (.eps, .pdf, .mps) and
% not a bitmapped formats (.jpeg, .png). IEEE frowns on bitmapped formats
% which can result in "jaggedy"/blurry rendering of lines and letters as
% well as large increases in file sizes.
%
% You can find documentation about the pdfTeX application at:
% http://www.tug.org/applications/pdftex

% correct bad hyphenation here
\hyphenation{}

\usepackage{cite}


\begin{document}
%
% paper title
% can use linebreaks \\ within to get better formatting as desired
\title{Recommending Functions in Spreadsheets from the Fuse Corpus}


% author names and affiliations
% use a multiple column layout for up to three different
% affiliations
\author{\IEEEauthorblockN{Shaown Sarker, Emerson Murphy-Hill}
\IEEEauthorblockA{Department of Computer Science\\
NC State University\\
Raleigh, North Carolina\\
Email: \{ssarker, emurph3\}@ncsu.edu}
}

% conference papers do not typically use \thanks and this command
% is locked out in conference mode. If really needed, such as for
% the acknowledgment of grants, issue a \IEEEoverridecommandlockouts
% after \documentclass

% for over three affiliations, or if they all won't fit within the width
% of the page, use this alternative format:
% 
%\author{\IEEEauthorblockN{Michael Shell\IEEEauthorrefmark{1},
%Homer Simpson\IEEEauthorrefmark{2},
%James Kirk\IEEEauthorrefmark{3}, 
%Montgomery Scott\IEEEauthorrefmark{3} and
%Eldon Tyrell\IEEEauthorrefmark{4}}
%\IEEEauthorblockA{\IEEEauthorrefmark{1}School of Electrical and Computer Engineering\\
%Georgia Institute of Technology,
%Atlanta, Georgia 30332--0250\\ Email: see http://www.michaelshell.org/contact.html}
%\IEEEauthorblockA{\IEEEauthorrefmark{2}Twentieth Century Fox, Springfield, USA\\
%Email: homer@thesimpsons.com}
%\IEEEauthorblockA{\IEEEauthorrefmark{3}Starfleet Academy, San Francisco, California 96678-2391\\
%Telephone: (800) 555--1212, Fax: (888) 555--1212}
%\IEEEauthorblockA{\IEEEauthorrefmark{4}Tyrell Inc., 123 Replicant Street, Los Angeles, California 90210--4321}}


% use for special paper notices
%\IEEEspecialpapernotice{(Invited Paper)}

% make the title area
\maketitle


\begin{abstract}
% TODO: should I refer to MS Excel instead of spreadsheets in certain places?
The most common form of end-user programming software is spreadsheets. Despite spreadsheets having a large array of functions built-in, most of these functions are underused. We look into modern recommender system technology to address this problem and present a collaborative filtering based function recommender system for spreadsheets. Our evaluation of the recommender system shows that the suggested functions are indeed helpful to real world spreadsheet users compared to a most popular algorithm based system. In this paper, we detail the methodology used to implement the system and the evaluation process. In addition, we outline a road-map to integrate the system as an effective tool in MS Excel.
\end{abstract}


% For peer review papers, you can put extra information on the cover
% page as needed:
% \ifCLASSOPTIONpeerreview
% \begin{center} \bfseries EDICS Category: 3-BBND \end{center}
% \fi
%
% For peerreview papers, this IEEEtran command inserts a page break and
% creates the second title. It will be ignored for other modes.
\IEEEpeerreviewmaketitle

\section{Introduction}

End-user programmers range from children to professionals including teachers, accountants, administrators, managers, and research scientists~\cite{ko2011state}. While many of these users are not formally trained software developers, the tasks they perform on a frequent basis are quite similar to that of a professional software engineer's.

Spreadsheets are the most common form of  end-user programming software. According to a study in 2005~\cite{scaffidi2005estimating}, nearly 23 million Americans use spreadsheets, constituting 30\% of the workforce. The use of spreadsheets is also very common in industry for analytical purposes. Winston~\cite{winston2001executive} estimates that around 90\% of all analysts in industry perform calculations in spreadsheets. From an end-users perspective, spreadsheet formulas can be viewed as fragments of source code as spreadsheet formulas can contain programming constructs like constants, variables, conditional statements, and references to other parts of the spreadsheet. It is important that the analysts using spreadsheet makes efficient use of the functionality available to them.

In Microsoft Excel, there are almost 350 unique functions, (472 in Microsoft Excel 2013 including the compatibility functions\footnote{https://support.office.com/en-us/article/Excel-functions-by-category-5f91f4e9-7b42-46d2-9bd1-63f26a86c0eb}). But, most of these are rarely used. To facilitate these very large number of spreadsheet analysts to use these functions, it is viable that we explore effective function recommendation in spreadsheets.

\texttt{Not sure how to tie in Fuse here.}

Researchers have been assembling and preserving spreadsheet corpora for the purpose of better understanding of end-user activities and designing tools to assist them~\cite{fisher2005euses, hermans2014enron}. In this paper, we look into Fuse~\cite{barik2015fuse}, the largest reproducible spreadsheet corpus known till date. We applied a slightly modified version of the user based collaborative filtering algorithm on the spreadsheets in Fuse to get function recommendations for an input spreadsheet. We evaluated the performance of our system against another commonly used recommender algorithm and found out that our collaborative filtering based recommendation system recommends functions in spreadsheet by \texttt{insert future calculated percentage here}. 

We first review previous work on recommendation systems in softwares and related work on making spreadsheet formula usage efficient (Section II). We then provide details of the spreadsheet corpus used for our system and the details of the algorithm used (Section III) before describing our evaluation measures (Section IV) and their outcome (Section V). We discuss the various aspects of our system (Section VI) along with the limitations and future works (Section VII) based on our contribution in this paper afterwards.

\section{Related Work}
Recommender systems have been applied with success in both academia and industry to help users cope with information overload and provide personalized suggestions~\cite{hill1995recommending, resnick1994grouplens,shardanand1995social}. Recommender systems assist users by distinguishing  products and services of interest to the user in a scenario when the number and diversity of options exceeds the user's decision making capability. 

Collaborative filtering is one of the most prominent recommending technologies available. The core concept of collaborative filtering is to apply a nearest neighbor method between a user's ratings or preferences and the preference of a large user community, and provide the user with recommendations by extrapolating based on how her selection or preference relate to that of the community. Collaborative filtering has been applied to recommend movies~\cite{miller2003movielens}, news~\cite{resnick1994grouplens}, books~\cite{linden2003amazon, liu2003adaptive}, research papers~\cite{mcnee2006don}, and school courses~\cite{farzan2006social, hsu2008personalized}. 

However, there has been little effort to use recommendation systems to help users of a large and complicated software system to learn and explore the functionalities and commands of the system. One of the notable attempts to use recommender systems to recommend command is the OWL~\cite{linton2000owl}, which makes command recommendations to Microsoft Word users. OWL recommends commands to an individual if she is not using certain commands at all but the community is using them on a frequent basis.

A similar approach to recommend commands in AutoCAD, a computer aided drafting software, called CommunityCommands has been developed by Matejka and colleagues based on collaborative filtering~\cite{matejka2009communitycommands}. The researchers of CommunityCommand showed that the collaborative filtering derived command recommendations were preferred by the AutoCAD users over the history based recommendation approach used in OWL.

\texttt{What should be the domain of the recommendations - the user or the spreadsheet}

In this paper, we build on these prior works by applying the concept of recommender systems in spreadsheet to recommend functions. We use the Fuse spreadsheet corpus as a source of function usage preference of the spreadsheet users' community and recommend functions contextualized to an individual spreadsheet by applying a slightly modified collaborative filtering algorithm.

\section{Methodology}

\subsection{Fuse Spreadsheet Corpus}

\subsection{Modified Collaborative Filtering}

\section{Evaluation}

%TODO possible subsections like automated evaluation (leave k-out evaluation and online evaluation)

\section{Results}

\section{Discussion}
%TODO consider naming it to contribution and/or merging it with Results secion.

\section{Limitations \& Future Work}

\section*{Acknowledgment}
%TODO acknowledgements if any (LAS information maybe?)


% trigger a \newpage just before the given reference
% number - used to balance the columns on the last page
% adjust value as needed - may need to be readjusted if
% the document is modified later
%\IEEEtriggeratref{8}
% The "triggered" command can be changed if desired:
%\IEEEtriggercmd{\enlargethispage{-5in}}

% references section

% can use a bibliography generated by BibTeX as a .bbl file
% BibTeX documentation can be easily obtained at:
% http://www.ctan.org/tex-archive/biblio/bibtex/contrib/doc/
% The IEEEtran BibTeX style support page is at:
% http://www.michaelshell.org/tex/ieeetran/bibtex/
\bibliographystyle{IEEEtran}
% argument is your BibTeX string definitions and bibliography database(s)
\bibliography{fuse_cf}
%
% <OR> manually copy in the resultant .bbl file
% set second argument of \begin to the number of references
% (used to reserve space for the reference number labels box)





% that's all folks
\end{document}
